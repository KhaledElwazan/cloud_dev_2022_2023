\documentclass{article}

\usepackage{xcolor}

\title{ Dockerizing a RESTful API for Managing Person Objects}
\author{}

\date{\textbf{\textcolor{red}{Deadline 2-5-2023, 23:45}}}

\begin{document}
\maketitle

\section*{Main Requirements}

You have been tasked with creating a Docker image that hosts a RESTful API for managing person objects. The API should allow users to perform CRUD (Create, Read, Update, Delete) operations on person objects.
Each person object should have the following attributes:

\begin{itemize}
	\item \texttt{id}: A unique identifier for the person object.
	\item \texttt{name}: The name of the person.
	\item \texttt{age}: The age of the person.
	\item \texttt{gender}: The gender of the person.
	\item \texttt{email}: The email address of the person.
\end{itemize}


\section*{API Specification}

The RESTful API should have the following endpoints:

\begin{itemize}
	\item \texttt{GET /persons}: Retrieve a list of all person objects.
	\item \texttt{POST /persons}: Create a new person object.
	\item \texttt{GET /persons/\{id\}}: Retrieve a specific person object by its ID.
	\item \texttt{PUT /persons/\{id\}}: Update a specific person object by its ID.
	\item \texttt{DELETE /persons/\{id\}}: Delete a specific person object by its ID.
\end{itemize}

\section*{Docker Image Specification}

You should write a Dockerfile that builds a Docker image containing the RESTful API code and its dependencies. You should then run a container from this Docker image to verify that the API is working correctly.
% To ensure that your solution works correctly, you should also create a test suite that verifies the functionality of the API. You can use an automated testing tool like \texttt{Mocha} or \texttt{Jest} to automate this process.
Your submission should include the following:

\begin{itemize}
	\item The Dockerfile used to build the Docker image.
	\item The RESTful API code.
	\item Documentation on how to run and test the Docker container locally.
	\item Public URL to the Docker image on Docker Hub.
\end{itemize}


\section*{Grading Criteria}

Your submission will be graded based on the following criteria:

\begin{itemize}
	\item The correctness of the RESTful API implementation.
	\item The correctness of the Dockerfile and the resulting Docker image.
	\item The correctness and completeness of the test suite.
	\item The clarity and completeness of the documentation.
\end{itemize}


% create a warning message
\begin{center}
	\begin{minipage}{0.9\textwidth}
		\begin{center}
			\textbf{Warning:} \textcolor{red}{The code you provide will be automatically tested for evaluation. Please make sure your code adheres to the specifications, complete, and works as expected before submitting.}
		\end{center}
	\end{minipage}
\end{center}
\end{document}
